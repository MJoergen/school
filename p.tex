\documentclass[12pt,oneside,a4paper]{article}

\usepackage[margin=1in]{geometry}
\usepackage[utf8]{inputenc} % Lærer LaTeX at forstå unicode - HUSK at filen skal
% være unicode (UTF-8), standard i Linux, ikke i
% Win.

\usepackage[danish]{babel} % Så der fx står Figur og ikke Figure, Resumé og ikke
% Abstract etc. (god at have).


\usepackage{graphicx}
\usepackage{amsfonts}
\usepackage{amsthm}        % Theorems
\usepackage{amsmath}
\usepackage{enumitem}
%\usepackage{hyperref}
\usepackage{float}
\usepackage{tcolorbox}


\newcommand{\bas}{\begin{eqnarray*}}
\newcommand{\eas}{\end{eqnarray*}}
\newcommand{\be}{\begin{equation}}
\newcommand{\ee}{\end{equation}}
\newcommand{\bea}{\begin{eqnarray}}
\newcommand{\eea}{\end{eqnarray}}

\theoremstyle{plain}
\newtheorem*{thm}{Sætning}
\newtheorem*{mydef}{Definition}
\newtheorem*{eks}{Eksempel}

\DeclareMathSymbol{,}{\mathord}{letters}{"3B}

\title{Opgaver til Wheatstone's bro}
\date{\vspace{-5ex}}

\begin{document}

\maketitle

\section*{Opgave 1}

På figuren er vist en Wheatstone bro, med en spændingskilde på $U=9 \,{\rm
V}$.  De tre af modstandene har værdierne $R_1 = 100 \,\Omega$, $R_2 = 200
\,\Omega$, $R_3 = 300 \,\Omega$. Den fjerde modstand $R_4$ har en værdi, som
gør, at spændingen mellem punkterne $J$ og $K$ er lig med nul. Broen er med
andre ord {\em balanceret}.

\begin{figure}[H]
    \centering
    \includegraphics[width=0.5\textwidth]{p1}
\end{figure}

\begin{enumerate}[label=(\alph*)]
    \item Hvilken værdi har resistoren $R_4$?
    \item Hvor stor er strømmen igennem $R_4$?
    \item Hvor stor er den samlede strøm, som spændingskilden leverer?
    \item Vil der ske nogen ændring i strømme eller spændinger i kredsløbet,
        hvis punkterne $J$ og $K$ forbindes med en ledning (og således
        kortsluttes)? Hvorfor eller hvorfor ikke?
\end{enumerate}

Nu erstattes resistoren $R_4$ med en anden resistor med værdien
$R_4=400\,\Omega$. Broen er derfor ikke længere balanceret.

\begin{enumerate}[label=(\alph*), resume]
    \item Hvad er spændingsforskellen mellem punkterne $J$ og $K$?
    \item Vil der ske nogen ændring i strømme eller spændinger i kredsløbet,
        hvis punkterne $J$ og $K$ forbindes med en ledning (og således
        kortsluttes)? Hvorfor eller hvorfor ikke?
\end{enumerate}

Nu kortsluttes punkter $J$ og $K$ -- stadig med den nye resistor $R_4$.
\begin{enumerate}[label=(\alph*), resume]
    \item Svær! Prøv om du kan beregne den strøm, som løber igennem
        kortslutningen. Husk, at strømme og spændingerne har ændret sig.
\end{enumerate}

\end{document}

